\documentclass[12pt]{article}
\usepackage{geometry}
\geometry{margin=1in}
\usepackage{fontspec}
\setmainfont{Times New Roman}
\usepackage{xeCJK}
\setCJKmainfont{PingFang SC}
\usepackage{hyperref}
\usepackage{graphicx}
\usepackage{float}
\usepackage{enumitem}
\usepackage{array}
\usepackage{longtable}
\usepackage{xcolor}
\hypersetup{
  colorlinks=true,
  linkcolor=black,
  urlcolor=blue
}

\title{cxresume 功能流程说明}
\author{Codex 自动整理}
\date{\today}

\begin{document}
\maketitle

\section{项目概览}
cxresume 是一个面向 Codex CLI 的会话恢复工具,核心能力是从历史日志中挑选或搜索会话,并调用 \texttt{codex resume <sessionId>} 继续对话。入口文件位于 \texttt{src/index.js},通过命令行参数决定运行模式,并协调配置读取、日志扫描、交互式界面与 Codex 启动流程。

\section{主要模块}
\begin{longtable}{>{\raggedright\arraybackslash}p{0.22\textwidth}p{0.72\textwidth}}
\textbf{配置管理} & \texttt{src/utils/config.js}:读取 \texttt{\~{}/.config/cxresume/config.json},合并命令行覆盖项,解析会话根目录。 \\
\textbf{日志发现} & \texttt{src/utils/sessionFinder.js}:基于 fast-glob 搜索 \texttt{*.jsonl},返回按修改时间倒序的文件列表。 \\
\textbf{元信息提取} & \texttt{src/utils/metaQuick.js}:流式读取日志首行或首个事件,快速拿到 \texttt{sessionId}、\texttt{cwd} 与起始时间。 \\
\textbf{日志解析与预览} & \texttt{src/utils/parser.js}、\texttt{src/utils/preview.js}:兼容新旧日志格式,提取对话消息并生成彩色预览文本。 \\
\textbf{交互界面} & \texttt{src/ui.js} 提供基于 Enquirer 的简易选择器;\texttt{src/tui/splitPicker.js} 提供 Blessed 分屏 TUI,支持分页、删除、复制和附加参数。 \\
\textbf{Codex 调用} & \texttt{src/utils/launch.js}:使用 \texttt{execaCommand} 在 shell 中运行构造好的 Codex 命令。 \\
\textbf{工作区会话管理} & \texttt{src/utils/workspaceSessions.js}:维护 \texttt{.cxresume\_sessions},必要时调用 Codex 创建新会话并记录。 \\
\end{longtable}

\section{CLI 主流程}
\begin{enumerate}[label=\arabic*.]
  \item 解析参数(\texttt{src/index.js:31})。处理 \texttt{cxresume cwd/.}、\texttt{--list}、\texttt{--search}、\texttt{--preview} 等开关。
  \item 判断是否输出帮助或版本,并在需要时直接返回。
  \item 读取配置并解析日志根目录(\texttt{src/index.js:127}),若目录不存在尝试旧版路径。
  \item 工作区模式下,读取或创建 \texttt{.cxresume\_sessions},并根据 ID 反查日志文件,准备预设列表(\texttt{src/index.js:173})。
  \item 根据不同入口执行:
  \begin{itemize}[leftmargin=2em]
    \item \textbf{列表模式}:打印最近 100 条会话信息。
    \item \textbf{直接打开}:使用 \texttt{--open} 指定的文件。
    \item \textbf{搜索模式}:全文检索命中后进入分屏 TUI。
    \item \textbf{交互模式}:默认进入分屏 TUI;若指定 \texttt{--legacy-ui} 则使用 AutoComplete 列表。
  \end{itemize}
  \item 在交互界面中,用户可浏览预览、附加参数、删除日志、复制 ID,或在工作区模式下新建记录。
  \item 选择目标会话后,读取元信息确认 \texttt{sessionId},可选打印最近消息作为预览。
  \item 组合命令 \texttt{codex resume <sessionId>},附加用户输入的额外参数;如指定 \texttt{--print} 或 \texttt{--no-launch} 则仅输出而不执行。
  \item 通过 \texttt{launchCodexRaw} 启动命令,继承当前终端输入输出,恢复 Codex 会话。
\end{enumerate}

\section{工作区流程补充}
当以 \texttt{cxresume cwd} 运行时:
\begin{itemize}[leftmargin=2em]
  \item 若 \texttt{.cxresume\_sessions} 不存在,工具会调用 \texttt{codex e ...} 启动一次新会话,解析输出中的 \texttt{session id} 并写入文件。
  \item \texttt{-n}/\texttt{--new} 会强制创建并立刻恢复一个全新的工作区会话。
  \item \texttt{-l}/\texttt{--latest} 则直接恢复最近一次记录的会话,不进入 TUI。
  \item 分屏 TUI 内部的 \texttt{s} 快捷键可即时触发“创建并恢复”流程。
\end{itemize}

\section{搜索与预览}
\begin{itemize}[leftmargin=2em]
  \item \textbf{搜索}:\texttt{src/utils/search.js} 对每个日志文件流式扫描消息内容,统计命中次数并按时间优先排序。
  \item \textbf{预览}:\texttt{selectRecentDialogMessages} 过滤最近若干条用户与助手对话,配合 TUI 中的彩色渲染,用于快速回忆上下文。
  \item \textbf{性能优化}:元信息、预览内容使用内存缓存,避免重复解析大型日志。
\end{itemize}

\section{流程图}
图~\ref{fig:flow} 展示了从命令行入口到 Codex 启动的关键流程,涵盖工作区分支与交互界面逻辑。

\begin{figure}[H]
  \centering
  \includegraphics[width=0.9\textwidth,height=0.9\textheight,keepaspectratio]{cxresume-flow-diagram.pdf}
  \caption{cxresume 主流程示意}
  \label{fig:flow}
\end{figure}

\section{关键交互要点}
\begin{itemize}[leftmargin=2em]
  \item TUI 在左侧列表显示会话元信息(ID、更新时间、工作目录、消息统计),右侧滚动展示会话片段。
  \item 支持快捷键:\texttt{↑/↓} 导航、\texttt{←/→} 翻页、\texttt{n} 在选中会话工作目录中创建新会话、\texttt{d} 删除会话文件、\texttt{-} 临时追加参数、\texttt{c} 复制 ID、\texttt{f} 切换全屏预览。
  \item 删除操作带确认弹窗,避免误删;工作区模式下对不属于当前目录的文件会提示并阻止恢复。
\end{itemize}

\section{扩展点与注意事项}
\begin{itemize}[leftmargin=2em]
  \item 预览默认关闭,可通过配置或 \texttt{--preview} 打开;注意大文件解析耗时。
  \item 如需自定义 Codex 命令或日志根目录,可在配置文件或命令行参数中覆盖。
  \item 工具假设日志目录结构与 Codex CLI 默认一致,若目录为空会给出提示信息。
\end{itemize}

\end{document}
